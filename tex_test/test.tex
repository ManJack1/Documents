%!TEX program = xelatex

\documentclass{article}
\usepackage{amsmath}
\usepackage{amssymb}
\usepackage{amsthm}
\usepackage{thmbox}
\usepackage{xeCJK}

% 定义有框的定理环境
\newtheorem[M]{boxedtheorem}{\indent 定理}[section]
\newtheorem[M]{boxedlemma}[boxedtheorem]{\indent 引理}
\newtheorem[M]{boxedproposition}[boxedtheorem]{\indent 命题}
\newtheorem[M]{boxedcorollary}[boxedtheorem]{\indent 推论}
\newtheorem[M]{boxeddefinition}[section]{\indent 定义}
\newtheorem[M]{boxedexample}[section]{\indent 例}
\newtheorem[M]{boxedremark}[section]{\indent 注}

% 定义没有框的解决方案环境
\newenvironment{solution}{\begin{proof}[\indent\bf 解]}{\end{proof}}

% 重新定义证明名称
\renewcommand{\proofname}{\indent\bf 证明}

\begin{document}

\section{简介}

\begin{boxedtheorem}[asdasdasd]
这是一个有框的定理。
\end{boxedtheorem}

\begin{boxedlemma}
这是一个有框的引理。
\end{boxedlemma}

\begin{boxedproposition}
这是一个有框的命题。
\end{boxedproposition}

\begin{boxedcorollary}
这是一个有框的推论。
\end{boxedcorollary}

\begin{boxeddefinition}
这是一个有框的定义。
\end{boxeddefinition}

\begin{boxedexample}
这是一个有框的例子。
\end{boxedexample}

\begin{boxedremark}
这是一个有框的注释。
\end{boxedremark}

\begin{solution}
这是一个没有框的解决方案。
\end{solution}

\end{document}

