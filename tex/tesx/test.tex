%!TEX program = xelatex
\documentclass[11pt, a4paper, oneside, UTF8]{ctexbook}
\usepackage{amsmath, amsthm, amssymb, bm, graphicx, hyperref, mathrsfs}
\usepackage[dvipsnames]{xcolor}
\usepackage{tikz}
\usetikzlibrary{backgrounds,arrows,shapes,tikzmark,calc}
\usepackage{geometry}
\usepackage{annotate-equations}
\usepackage{extarrows}
\usepackage{thmbox}

% Custom environments and commands
\newenvironment{note}
{\par\textcolor{blue}{\bfseries Note:}\itshape}{\par}
\newenvironment{remark}
{\par\textcolor{blue}{\bfseries Remark:}\itshape}{\par}
\newtheorem[M]{theorem}{Theorem}[section]
\newtheorem[M]{lemma}[theorem]{Lemma}
\newtheorem[M]{definition}{Definition}[section]
\renewcommand{\eqnannotationfont}{\bfseries\small}

\title{{\Huge{\textbf{Linear Algebra}}}\\------HOFFMAN AND RAY KUNZE}
\author{ManJack}
\date{\today}
\linespread{1.5}

\geometry{
  a4paper,
  total={170mm,257mm},
  left=20mm,
  top=20mm,
}

\begin{document}

\maketitle

\pagenumbering{roman}
\setcounter{page}{1}

\newpage
\begin{center}
  \Huge\textbf{前言}
\end{center}

数据分析中方差分析的实验报告
\begin{flushright}
  \begin{tabular}{c}
    ManJack \\
    \today
  \end{tabular}
\end{flushright}

\newpage
\tableofcontents
\newpage
\pagenumbering{arabic}
\setcounter{page}{1}

\chapter{Linear Equations}

% Start your content here
对于总体$X_i$的样本$X_{ij} \quad i = 1,2,\cdots,k;j = 1,2,\cdots,n_i$,有\[
\overline{X} = \frac{1}{n}\sum_{i=1}^k\sum_{j=1}^{n_i}X_{ij}, \quad \overline{x}_i = \frac{1}{n_i}\sum_{j=1}^{n_i}X_{ij}, \quad \overline{\overline{X}} = \frac{1}{k}\sum_{i=1}^k\overline{x}_i\]
\[
 S_t = \sum_{i=1}^k\sum_{j=1}^{n_i}(X_{ij} - \overline{X})^2,S_e = \sum_{i=1}^k\sum_{j=1}^{n_i}(X_{ij} - \overline{x}_i)^2,S_A = \sum_{i=1}^k n_i(\overline{x}_i - \overline{\overline{X}})^2
\]

在平方和的基础上,可以得到平方和的期望\[
  E(S_e) = (n-k)\sigma^2, \quad E(S_A) = (k-1)\sigma^2, \quad E(S_t) = n\sigma^2
\]
\[
  S_A = \sum_{i=1}^k\frac{(\overline{x}_i - \overline{\overline{X}})^2}{\sigma^2}, \quad S_e = \sum_{i=1}^k\sum_{j=1}^{n_i}\frac{(X_{ij} - \overline{x}_i)^2}{\sigma^2}
.\]
于是\[
  ES_A = \sum\limits_{i=1}^{r}n_iE(\overline{X}_i^2) - nE(\overline{\overline{X}}^2) = \sum\limits_{i=1}^{r}(\frac{\sigma}{n_i}+\mu^2)n_i - n(\frac{\sigma}{n}+\mu^2) = \sigma^2\sum\limits_{i=1}^{r}\frac{n_i}{n} = (r-1)\sigma^2 +\sum\limits_{i=1}^{r}n_i\alpha^2
.\]

于是可得
\[
  E(\frac{S_e}{n-r})= \sigma^2, \quad E(\frac{S_A}{r-1}) = \sigma^2 + \frac{1}{r-1}\sum\limits_{i=1}^{r}n_i\alpha^2
\]

因此不论,$H_0$是否成立,都有\[
  E(\frac{S_e}{n-r}) = E(\frac{S_A}{r-1}) = \sigma^2
\]都是$\sigma^2$的无偏估计量,因此\[
  \frac{S_e}{\sigma^2} \sim \chi^2(n-r), \quad \frac{S_A}{\sigma^2} \sim \chi^2(r-1)
\]
因此我们可以给定判断显著性水平
当F统计量$F_0 = \frac{S_A/(r-1)}{S_e/(n-r)}$的值大于$F_{\alpha}(r-1,n-r)$时,拒绝$H_0$,否则不拒绝$H_0$.其中$F_{\alpha}(r-1,n-r)$是F分布的上$\alpha$分位数.
当计算F时我们还可以利用以下公式来简化计算
若令
\[
R = \sum\limits_{i=1}^{r}\sum\limits_{j=1}^{n_i}(X_ij)^2,K = \sum\limits_{i=1}^{r}\sum\limits_{j=1}^{n_i}X_{ij},Q = \sum\limits_{i=1}^{r}n_i\overline{X}_i^2
.\]

则有\[
  S_t = R - \frac{K^2}{n}, \quad S_A = \frac{K^2}{n} - Q, \quad S_e = R - Q
\]
参数估计
1.总体均值$\mu$的点估计
\[
 \vec{\mu} = \overline{X}_i, \quad \vec{\mu}_i = \overbar{x}_i,i = 1,2,\cdots,r,\quad \vec{\sigma}^2 = \frac{S_e}{n-r}
\]
\end{document}
