%!TEX program = xelatex
\documentclass[11pt, a4paper, oneside, UTF8]{ctexbook}
\usepackage{amsmath, amsthm, amssymb, bm, graphicx, hyperref, mathrsfs}
\usepackage[dvipsnames]{xcolor}
\usepackage{tikz}
\usetikzlibrary{backgrounds,arrows,shapes,tikzmark,calc}
\usepackage{geometry}
\usepackage{annotate-equations}
\usepackage{extarrows}
\usepackage{thmbox}

% Custom environments and commands
\newenvironment{note}
{\par\textcolor{blue}{\bfseries Note:}\itshape}{\par}
\newenvironment{remark}
{\par\textcolor{blue}{\bfseries Remark:}\itshape}{\par}
\newtheorem[M]{theorem}{Theorem}[section]
\newtheorem[M]{lemma}[theorem]{Lemma}
\newtheorem[M]{definition}{Definition}[section]
\renewcommand{\eqnannotationfont}{\bfseries\small}

\title{{\Huge{\textbf{Linear Algebra}}}\\------HOFFMAN AND RAY KUNZE}
\author{ManJack}
\date{\today}
\linespread{1.5}

\geometry{
  a4paper,
  total={170mm,257mm},
  left=20mm,
  top=20mm,
}

\begin{document}

\maketitle

\pagenumbering{roman}
\setcounter{page}{1}

\newpage
\begin{center}
  \Huge\textbf{前言}
\end{center}

这是数学系线性代数的笔记,写给自己。如有错误请见谅,这些只是作为分享。

\begin{flushright}
  \begin{tabular}{c}
    ManJack \\
    \today
  \end{tabular}
\end{flushright}

\newpage
\tableofcontents
\newpage
\pagenumbering{arabic}
\setcounter{page}{1}

\chapter{Linear Equations}

% Start your content here
在单因素r个水平试验中,设因素A的r个水平为$A_1,A_2,\cdots,A_r$,A的第i个水平重复$n_i$次,记$N=n_1+n_2+\cdots+n_r$,$x_{ij}$表示第i个水平的第j次试验的结果,$i=1,2,\cdots,r;j=1,2,\cdots,n_i$,则称随机变量$x_{ij}$服从r个水平的因素A的单因素试验模型,记为$A_{r}(n_1,n_2,\cdots,n_r)$,其数学模型为 $x_{ij}=\mu_i+\varepsilon_{ij}$,其中$\mu_i$为第i个水平的总体均值,$\varepsilon_{ij}$为第i个水平的第j次试验的随机误差,假定$\varepsilon_{ij}$相互独立,且服从均值为0,方差为$\sigma^2$的正态分布,即$\varepsilon_{ij}\sim N(0,\sigma^2)$,$i=1,2,\cdots,r;j=1,2,\cdots,n_i$。

$n = s$



\end{document}
