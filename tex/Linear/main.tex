%!TEX program = xelatex
\documentclass[11pt, a4paper, oneside,UTF8]{ctexbook}
\usepackage{amsmath, amsthm, amssymb, bm, graphicx, hyperref, mathrsfs}
\usepackage[dvipsnames]{xcolor}
\usepackage{tikz}
\usetikzlibrary{backgrounds}
\usetikzlibrary{arrows,shapes}
\usetikzlibrary{tikzmark}
\usetikzlibrary{calc}
\usepackage{graphicx}
\usepackage{geometry}
\usepackage{annotate-equations}
\usepackage{extarrows}


%setting annotate
\renewcommand{\eqnannotationfont}{\bfseries\small}
\usepackage{titlesec}
\titleformat{\section}[hang]{\normalfont\Large\bfseries}{\thesection}{1em}{}
\titlespacing{\section}{0pt}{\baselineskip}{0.5\baselineskip}

\geometry{
	a4paper,
	total={170mm,257mm},
	left=20mm,
	top=20mm,
}



%colorbox
\newcommand{\hlmath}[2]{\colorbox{#1!17}{$\displaystyle #2$}}
%\newcommand{\highlight}[2]{\colorbox{#1!17}{$#2$}}
\newcommand{\hltext}[2]{\colorbox{#1!47}{$\displaystyle #2$}}



%setting mathenv
\newtheorem{theorem}{\indent 定理}[section]
\newtheorem{lemma}[theorem]{\indent 引理}
\newtheorem{proposition}[theorem]{\indent 命题}
\newtheorem{corollary}[theorem]{\indent 推论}
\newtheorem{definition}{\indent 定义}[section]
\newtheorem{example}{\indent 例}[section]
\newtheorem{remark}{\indent 注}[section]
\newenvironment{solution}{\begin{proof}[\indent\bf 解]}{\end{proof}}
\renewcommand{\proofname}{\indent\bf 证明}


\title{{\Huge{\textbf{Linear Algebra}}}\\------HOFFMAN AND RAY KUNZE}
\author{ManJack}
\date{\today}
\linespread{1.5}


\begin{document}

\maketitle

\pagenumbering{roman}
\setcounter{page}{1}
\newpage
\begin{center}
	\Huge\textbf{前言}
\end{center}~\

这是数学系线      性代数的笔记,写给自己。如有错误请见谅,这些只是作为分享。
~\\
\begin{flushright}
	\begin{tabular}{c}
		ManJack \\
	\end{tabular}
\end{flushright}

\newpage
\pagenumbering{Roman}
\setcounter{page}{1}
\today
\tableofcontents
\newpage
\setcounter{page}{1}
\pagenumbering{arabic}
\chapter{Linear Equations}
\section{环和域}
\subsection{加群和环的定义}
\begin{definition}[\textbf{加群}]
	假如一个\textbf{Abel群}的代数运算为加法,并且用符号‘+’表示,则该群叫做\textbf{加群}。

\end{definition}

\begin{remark}
	加群的单位元e是唯一的,且e=0。称作\textbf{零元},我们有以下的计算规则:
	\[
		0+a=a+0=a
	\]
\end{remark}

\begin{definition}[\textbf{环}]
	一个集合R称之为环满足:
	\begin{enumerate}
		\item R是一个加群
		\item R对一个乘法来说是一个半群(半群是一个群胚+结合律)
		\item 在集合R上,乘法对加法满足分配率a(b+c)=(ab+ac)
	\end{enumerate}
\end{definition}

\subsection{交换元、单位元、零因子、整环}
\begin{definition}[\textbf{交换环}]
	一个环叫做一个交换环,假如:\[
		ab = ba
	\]
\end{definition}

在环中乘法运算下的单位元,叫做环的单位元。
\begin{definition}[\textbf{单位元}]
	环中的单位元e,假如对于R的任意元素a来说,有:


	\[
		\eqnmarkbox[red]{e_1}{e}a = a \eqnmarkbox[red]{e_2}{e} = a
	\]

	\annotate[yshift=-0.5em]{below}{e_1,e_2}{单位元}


\end{definition}


\begin{definition}[零因子]
	一个环中的两个元素a,b之间如果有一个是0,那么ab=0.
	但反之不成立.\[
		ab = 0 \xLongrightarrow{\textbf{不成立}}a=0 \ or \ b = 0
	\]
\end{definition}

\begin{example}
	例如模n的\textbf{剩余类环}:假设n=ab\\
	若n不是素数,假设:\[
		[\alpha]\neq  [0],[b]\neq  [0],[\alpha][b]=[\alpha b]=[n]=[0]
	\]
	则我们可以得知$ ab = 0 \xLongrightarrow{\textbf{不成立}}a=0 \ or \ b = 0
	$
\end{example}
\begin{remark}
	若是在一个环里,
	\[
		a\leq 0,b\neq 0,ab=0
	\]则a被称为\textbf{左零因子},b被称为\textbf{右零因子}
\end{remark}
\begin{definition}[整环]
	一个环叫做\textbf{整环},满足:
	\begin{enumerate}
		\item 乘法交换律:\[
			      ab = ba
		      \]
		\item R有单位元1:\[
			      1a=a1=a
		      \]
		\item R没有零因子:\[
			      ab=0 \xLongrightarrow{}  a = 0\ or \ b=0
		      \]
	\end{enumerate}
\end{definition}
\begin{remark}
	a,b可以是任意R中的元.
\end{remark}
\subsection{除环、域}

\begin{definition}[除环]
	一个环R叫做一个除环,满足:
	\begin{enumerate}
		\item R至少包含一个不为0的元
		\item R有一个单位元
		\item R的每个不等于0的元有一个逆元
	\end{enumerate}
\end{definition}
\begin{definition}[除环]


	一个集合 $F$ 被称为域,如果满足以下条件:
	\begin{enumerate}
		\item 加法封闭性:$\forall a, b \in F$,有 $a + b \in F$。
		\item 加法可交换性:$\forall a, b \in F$,有 $a + b = b + a$。
		\item 加法单位元素:存在加法单位元素 0,使得 $\forall a \in F$,有 $a + 0 = a$。
		\item 加法逆元素:$\forall a \in F$,存在加法逆元素 $-a$,使得 $a + (-a) = 0$。
		\item 乘法封闭性:$\forall a, b \in F$,有 $a \cdot b \in F$。
		\item 乘法可交换性:$\forall a, b \in F$,有 $a \cdot b = b \cdot a$。
		\item 乘法单位元素:存在乘法单位元素 1,使得 $\forall a \in F$,有 $a \cdot 1 = a$。
		\item 乘法逆元素:$\forall a \in F$,对于非零元素,存在乘法逆元素 $a^{-1}$,使得 $a \cdot a^{-1} = 1$。
		\item 分配律:$\forall a, b, c \in F$,满足 $(a + b) \cdot c = a \cdot c + b \cdot c$。
	\end{enumerate}

\end{definition}

\begin{definition}[Subfiled]

	设 $F$ 是一个域。如果 $K \subseteq F$ 满足以下条件,则称 $K$ 是 $F$ 的\emph{子域}:
	\begin{enumerate}
		\item $K$ 非空,并且包含域 $F$ 中的加法单位元素 0 和乘法单位元素 1。
		\item 对于任意的 $a$ 和 $b$ 属于 $K$,$a + b$ 和 $a \cdot b$ 也都属于 $K$(其中 $+$ 和 $\cdot$ 分别表示域 $F$ 中的加法和乘法运算)。
		\item 对于任意的 $a$ 属于 $K$,它的相反元素 $-a$ 也属于 $K$。
		\item 对于任意的非零元素 $a$ 属于 $K$,它的乘法逆元素 $a^{-1}$ 也属于 $K$。
	\end{enumerate}
\end{definition}


\begin{definition}[Characteristic]
	In abstract algebra, "characteristic" is an important concept for a ring or a field. The characteristic is used to describe the smallest positive integer $n$ for which $n$ times the multiplicative identity $1$ equals the additive identity (usually denoted as $0$) in the algebraic structure.

	For a ring (a set with addition and multiplication operations, satisfying certain algebraic rules), the characteristic refers to the smallest positive integer $n$ such that $n$ times $1$ equals $0$ (or defined as $0$ if there is no such $n$).

	For a field (a special type of ring where every non-zero element has a multiplicative inverse), the characteristic is also a positive integer $n$ or zero, representing $n$ times $1$ equals $0$ or having characteristic zero if there is no such $n$.

	The significance of the characteristic lies in its impact on the properties and structure of the ring or field. Particularly, in the case of a field, the characteristic is either a prime number or zero. This distinction is useful as it allows us to differentiate between fields of different characteristics and has important applications in properties of algebraic equations and polynomials.
\end{definition}


\end{document}
