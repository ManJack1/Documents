\documentclass[12pt, a4paper,oneside, UTF8]{ctexbook}
\usepackage[dvipsnames]{xcolor}
\usepackage{amsmath}   % 数学公式
\usepackage{amsthm}    % 定理环境
\usepackage{amssymb}   % 更多公式符号
\usepackage{graphicx}  % 插图
\usepackage[colorlinks,linkcolor=black]{hyperref} % 修改超链接颜色
\usepackage{mathrsfs}  % 数学字体
\usepackage{enumitem}  % 列表
\usepackage{xcolor}    % 颜色
\usepackage{geometry}  % 页面调整

\graphicspath{ {flg/},{../flg/} }            % 配置图形文件检索目录

% 定理环境
\theoremstyle{definition}                    % 设置环境样式
\newtheorem{defn}{\indent 定义}[section]      

\newtheorem{lemma}{\indent 引理}[section]     % 引理 定理 推论 准则 共用一个编号计数
\newtheorem{thm}[lemma]{\indent 定理}
\newtheorem{corollary}[lemma]{\indent 推论}
\newtheorem{criterion}[lemma]{\indent 准则}

\newtheorem{proposition}{\indent 命题}[section]
% \newtheorem{example}{\indent 例}[section]
\newtheorem{example}{\indent \color{SeaGreen}{例}}[section]     % 绿色的 例,不需要使用上面注释的
\newtheorem*{rmk}{\indent 注}

% 两种方式定义中文的 证明 和 解 的环境,

% 缺点:\qedhere 命令将会失效【技术有限,暂时无法解决】
%\renewenvironment{proof}{\par\textbf{证明.}\;}{\qed\par}
%\newenvironment{solution}{\par{\textbf{解.}}\;}{\qed\par}

% 缺点:\bf 是过时命令,可以用 textb f等替代,但编译会有关于字体的警告,不过不影响使用【技术有限,暂时无法解决】
\renewcommand{\proofname}{\indent\bf 证明}
\newenvironment{solution}{\begin{proof}[\indent\bf 解]}{\end{proof}}

% 设置页面页边距
\geometry{top=25.4mm,bottom=25.4mm,left=20mm,right=20mm,headheight=2.17cm,headsep=4mm,footskip=12mm}

% 统一行间公式与上下文之间的间距
\makeatletter
\renewcommand\normalsize{%
    \@setfontsize\normalsize{12}{15.5}%
    \abovedisplayskip 6\p@ \@plus3\p@ \@minus7\p@
    \abovedisplayshortskip \z@ \@plus3\p@
    \belowdisplayshortskip 4\p@ \@plus3.5\p@ \@minus3\p@
    \belowdisplayskip \abovedisplayskip
    \let\@listi\@listI}
\makeatletter

% 设置列表环境的上下间距
\setenumerate[1]{itemsep=5pt,partopsep=0pt,parsep=\parskip,topsep=5pt}
\setitemize[1]{itemsep=5pt,partopsep=0pt,parsep=\parskip,topsep=5pt}
\setdescription{itemsep=5pt,partopsep=0pt,parsep=\parskip,topsep=5pt}

\linespread{1.6}
\def\d{\textup{d}}